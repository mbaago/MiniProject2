\section{Sentiment analysis}

\subsection{Steps in constructing a sentiment classifier}
The overall structure in a sentiment classifier is:
\begin{enumerate}
    \item Tokenization.
    \item Feature extraction.
    \item Classification, using e.g. Naïve Bayes.
\end{enumerate}
%
When classifying a review, several problems arise, such as:
\begin{itemize}
    \item Markup (e.g. bold text)
    \item Capitalization, indicating shouting/a stronger opinion.
    \item Dates, addresses, ...
    \item Emoticons, adding to a positive/negative value.
    \item Masked swearing, e.g. ****, f**k, \@\#!\%, ...
    \item Lengthening of words such as really: reaaaaaally.
    \item Negations, especially when early in a sentence.
\end{itemize}

To handle this, we use Potts's sentiment aware tokenizer: \url{http://sentiment.christopherpotts.net/code-data/happyfuntokenizing.py}. Negations are handled by adding \texttt{\_NEG} to all words following the negations, until the first clause-level punctuation mark.



% what did we do
% corner cutting


\subsection{How good is the classifier}